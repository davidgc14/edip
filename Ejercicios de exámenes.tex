\documentclass[fleqn]{article}

%\pgfplotsset{compat=1.17}

\usepackage{mathexam}
\usepackage{amsmath}
\usepackage{amsfonts}
\usepackage{graphicx}
\usepackage{subcaption}
\usepackage{systeme}
\usepackage{microtype}
\usepackage{multirow}
\usepackage{pgfplots}
\usepackage{listings}
\usepackage{tikz}
\usepackage{verbatim} %comentarios de párrafos
\usepackage{dsfont} %Numeros reales, naturales...
\usepackage{cancel}
\usepackage{hyperref}
\usepackage[spanish]{babel}
\usepackage[utf8]{inputenc} %Indice


%\graphicspath{{images/}}
\newcommand*{\QED}{\hfill\ensuremath{\square}}



%Estructura de ecuaciones
%\setlength{\textwidth}{15cm} \setlength{\oddsidemargin}{5mm}
%\setlength{\textheight}{23cm} \setlength{\topmargin}{-1cm}


\author{David García Curbelo}
%\title{}

\pagestyle{empty}


\def\R{\mathds{R}}
\def\Z{\mathds{Z}}
\def\N{\mathds{N}}
\def\X{\mathbf{X}}

\def\sup{$^2$}

\def\next{\quad \Rightarrow \quad}

\def\s{\thinspace}

\def\ss{\thinspace \thinspace}

\begin{document}
        \setcounter{page}{1}
        \pagestyle{plain}



        \begin{titlepage}
                \centering
                {\bfseries\LARGE Universidad de Granada \par}
                \vspace{1cm}
                {\scshape\Large Facultad de Ciencias \par}
                \vspace{5cm}
                {\scshape\Huge Ejercicios de exámenes Resueltos \par}
                \vspace{3cm}
                {\itshape\Large Estadística descriptiva en introd. a la Probabilidad \par}
                \vfill
                {\textit\Large AmigoCanario \par}
        \end{titlepage}


        \textit{En el presente documento se presenta un compendio de todas las cuestiones del marco teórico que se han preguntado en los
        últimos 10 años, con la intención de que sirva al lector a modo de guía teórica para los exámenes venideros. Como podréis 
        observar, no hay preguntas en exceso, y éstas mismas que aquí os presento se han repetido convocatoria tras convocatoria 
        a lo largo de los años. Espero que te sirva de ayuda tanto como me ha servido a mí, y espero que hagas buen uso del mismo.}
        
        \tableofcontents

        \newpage

        \section{Estadística Unidimensional y Bidimensional}

        \begin{enumerate}
                
                \subsection{Opción Múltiple}

                \item \textbf{Responde correctamente a las siguientes cuestiones:}
                        \begin{enumerate}
                                \item \textbf{Si el coeficiente de variación de Pearson de $X$ es $CV(X) = 0.32$ y la variable $Y = \frac{X+4}{2}$, entonces:}
                                        \begin{enumerate}
                                                \item $CV(Y) = 2.16$
                                                \item Las variables $X$ e $Y$ tienen la misma dispersión relativa.
                                                \item $CV(Y) = \frac{0.32}{1+\frac{4}{\bar{x}}}$
                                                \item Ninguna de las anteriores.
                                        \end{enumerate}

                                \item \textbf{La razón de correlación:}
                                        \begin{enumerate}
                                                \item Tiene por rango de variación $[-1, 1]$.
                                                \item En el caso del ajuste lineal coincide con el coeficiente de correlación lineal.
                                                \item Indica la proporción de la varianza de la variable dependiente debida a la varianza residual. 
                                                \item Indica la proporción de la varianza de la variable dependiente, explicada por la regresión. 
                                        \end{enumerate}

                                \item \textbf{El valor de la vivienda se ha incrementado un 10\%, 3\%, 2\% y 9\% respectivamente durante los últimos 
                                        cuatro años. El incremento medio anual del valor de la vivienda durante dicho periodo ha sido:}
                                        \begin{enumerate}
                                                \item 25.97\%
                                                \item 5.94\%
                                                \item 6\%
                                                \item 4.82\%
                                        \end{enumerate}

                                \item \textbf{A partir de la siguiente distribución:}
                                        \begin{center}
                                                \begin{tabular}{c|c|c}
                                                        $x_i$ & $f_i$ & $N_i$ \\
                                                        \hline
                                                        0 & 0.5 & \\
                                                        \hline
                                                        1 & 0.2 & 7 \\
                                                        \hline
                                                        2 & 0.3 & \\
                                                \end{tabular}
                                        \end{center}
                                        \textbf{se puede afirmar que:}
                                        \begin{enumerate}
                                                \item La media aritmética es 0.5.
                                                \item El valor 1 se repite 7 veces.
                                                \item Existe un total de 10 observaciones.
                                                \item El valor mediano es $x_2$.
                                        \end{enumerate}
                                
                                 \item \textbf{La varianza de los residuos de la curva de regresión de $Y$ sobre $X$ vale 1 y la varianza de $Y$ vale 4.
                                        Si $Y' = 1-Y$, entonces:}  
                                        \begin{enumerate}
                                                \item La razón de correlación de $Y'$ sobre $X$ es mayor que 0.75.
                                                \item El coeficiente de determinación lineal de $X$ e $Y'$ vale 0.75.
                                                \item El error cuadrático medio asociado a la curva de regresión de $Y'$ sobre $X$ vale 1.
                                                \item $\eta_{y'/x}^2 = 1 - \eta_{y/x}^2$.
                                        \end{enumerate} 

                                \item \textbf{Sea $(X,Y)$ una variable estadística bidimensional. Ajustar por mínimos cuadrados un modelo de la forma 
                                        $y= ax^2$ e identificar la afirmación correcta:}
                                        \begin{enumerate}
                                                \item La medida de los residuos es nula.
                                                \item $a= m_{21} / m_{20}$.
                                                \item La varianza de los residuos coincide con el error cuadrático medio.
                                                \item $a= m_{21} / m_{40}$.
                                        \end{enumerate}

                                \item \textbf{Si $\bar{x} > 1/2$ e $Y = 2X -1$, entonces:}
                                        \begin{enumerate}
                                                \item $Y$ tiene menos dispersión absoluta respecto a su media que $X$.
                                                \item La media de $Y$ es más representativa que la media de $X$.
                                                \item Las medias de $X$ y de $Y$ son igual de representativas.
                                                \item La media de $X$ es más representativa que la media de $Y$.
                                        \end{enumerate} 
                                
                        \end{enumerate}


                \newpage

                \subsubsection{Soluciones}

                        \begin{enumerate}
                                \item Solución apartado 3). Como queremos calcular el $CV(Y)$, tenemos que aplicar el cambio de variable 
                                        de $Y = \frac{X}{2} + 2$, lo que es lo mismo, todos sus valores en la nueva variable quedan modificados tal que 
                                        $y_i = \frac{x_i}{2} + 2$. Como sabemos, 
                                        $$CV_X = \frac{\sigma_x}{\bar{x}} \next CV_{aX+b} = \frac{a\sigma_x}{a\bar{x} + b} \next CV_Y = \frac{\frac{1}{2}\sigma_x}{\frac{\bar{x}}{2} + 2}$$
                                        Ahora diviendo entre la media $\bar{x}$ el numerador tenemos
                                        $$CV_Y = \frac{\frac{1}{2}\sigma_x}{\frac{\bar{x}}{2} + 2} = \frac{\frac{1}{2}\frac{\sigma_x}{\bar{x}}}{\frac{1}{2} + \frac{2}{\bar{x}}} = 
                                        \frac{1}{2}\frac{CV_X}{\frac{1}{2} + \frac{2}{\bar{x}}} = \frac{CV_X}{1 + \frac{4}{\bar{x}}}$$ \\

                                \item Solución apartado 4). Aquí hay poco que explicar. El primero no puede ser, ya que se trata de un
                                        intervalo abierto, y la razón de correlación puede tomar valor 0 (ajuste perfecto). Como la fórmula es
                                        $$\eta^2_{y/x} = 1 - \frac{\sigma_{ry}^2}{\sigma_y^2}$$
                                        Vemos que indica la proporción de la variable dependiente ($y$) con respecto a la regresión.\\

                                \item Solución apartado 2). Para este ejercicio debemos tener en cuneta que se trata de \textit{incremento}, es decir,
                                        en realidad estamos trabajando con los porcentajes 110\%, 103\%, 102\% y 109\%. Al tratarse de porcentajes, realizamos su media 
                                        geométrica, y obtenemos así 
                                        $$I_M = \sqrt[n]{\prod_{i=1}^n x_i} = \sqrt[4]{1.1 \cdot 1.03 \cdot 1.02 \cdot 1.09} = 1,0594 $$
                                        $$\next \text{Incremento del }5.94\%$$\\

                                \item Solución apartado 3). Si calculamos la media veremos que su valor es $\bar{x} = 0.8 \neq 0.5$. El valor 
                                        mediano no puede ser 1 ya que el 50\% de los datos están en $x_1$ ($f_1 = 0.5$). Si completamos la tabla, vemos que 
                                        $F_2 = 0.7$, y como $n = \frac{N_i}{F_i}$, tenemos que $n = 10$.\\   

                                \item Solución apartado 3). Sabemos por hipótesis que $\sigma_y^2 = 4$ y que $\sigma_{ry}^2 = ECM(Y/X) = 1$. Como sabemos que la varianza
                                        sólo se ve afectada por traslaciones, al tener $\sigma_y^2 = \sigma_{ey}^2 + \sigma_{ry}^2$ dicha traslación afecta de igual manera
                                        a las otras dos varianzas del lado derecho de la igualdad. Así tenemos $ECM(Y'/X) = \sigma_{ry'}^2 = (-1)^2\sigma_{ry}^2 = 1$. \\
                                
                                \item Solución apartado 4). Tomando un polinomio de grado arbitrario, de la forma $y = a_0 + a_1 x + ... + a_n x^n$, vemos que el único término
                                        no nulo es $a_2 = a$. Igualando las derivadas parciales a cero de la función $ \phi (a_2) = \sum_{i=1}^k \sum_{j=1}^p f_{ij} [y_j - ax^2]^2$
                                        obtenemos 
                                        $$m_{n,1} = \sum_{i=0}^n a_i m_{n+i,0}$$
                                        Como $n=2$ y $a_0 = a_1 = 0$, tenemos $m_{2,1} = a_2 m_{4,0}$. Despejando obtenemos $a = \frac{m_{2,1}}{m_{4,0}}$.\\
                        \end{enumerate}


                \newpage
                
                \subsection{Verdadero/Falso}


                \item \textbf{Justificar la veracidad o falsedad de las siguientes afirmaciones:}
                
                        \begin{enumerate}
                                \item El cuartil de orden 2 de la distribución de una variable estadística continua no siempre 
                                        coincide con su mediana.
                                \item El coeficiente de determincación, en el caso de la regresión lineal, coincie con el 
                                        coeficiente de correlación lineal.
                                \item Sea $X$ una variable estadística. Sea $Y = 3X - 2$. Entonces 
                                        $CV(Y) = \frac{3CV(X)}{3 - \frac{2}{\bar{x}}}$.
                                \item Si $\mu_{11} = 0$ entonces las variables son independientes (momento conjunto no centrado).
                                \item Si las variables $X$ e $Y$ son independientes, entonces los momentos conjuntos no centrados
                                        serán todos nulos.
                                \item En la curva de regresión mínimo cuadrática $Y/X$, los residuos tienen media $\bar{Y}$ y los 
                                        valores ajustados tienen media 0.
                                \item Si $X$ e $Y$ son independientes, entonces la covarianza de las variables es nula.
                                \item Si $X$ e $Y$ son independientes, entonces la razón de correlación de $Y$ sobre $X$ es mayor que
                                        el coeficiente de determincación lineal.
                                \item Si $X$ e $Y$ son independientes, entonces la media de los residuos lineales mínimo cuadráticos 
                                        de $Y/X$ es $\bar{Y}$.
                                \item Si $X$ e $Y$ son independientes, entonces la varianza de los residuos lineales mínimo cuadráticos
                                        de $Y/X$ es $\sigma_y ^2$.
                                \item Si todos los valores de $X$ están comprendidos en un intervalo de amplitud 2, la varianza de $X$ es 
                                        menor o igual que 4.
                                \item Si la varianza de $X$ vale 4 y el error cuadrático medio asociado a la recta de regresión de $X$
                                        sobre $Y$ vale 3, la razón de correlación de $X$ sobre $Y$ es menor que 0.25.
                                \item Si $X$ es independiente de $Y$, entonces las medias condicionadas de $Y$ a los distintos valores 
                                        de $X$ tienen varianza 0.
                                \item si $X' = -3X + 1$ e $ Y' = 3Y - 2$, entonces $R_{x'y'} = R_{xy}$.
                                \item Si $X$ e $Y$son variables independientes tales que $\bar{x} = -3$ y $m_{11} = 6$, entonces la media 
                                        de $Y' = -3Y + 1$ vale 7.
                                \item Si las rectas de regresión son $4x + y = 1$ y $5x + y = 2$, entonces $\bar{x} = 1$.
                        \end{enumerate}


                \newpage

                \subsubsection{Soluciones}

                        \begin{enumerate}
                                \item Falso. $P_{50} = D_{10} = Q_2 = Me$\\
                                \item Falso. Sólo coinciden en en caso en el que las rectas son crecientes y que además exista dependencia 
                                                lineal recíproca entre $X$ e $Y$ ($R_{xy} = R_{xy}^2 = 1$) \\
                                \item Verdadero. Como queremos calcular el $CV(Y)$, tenemos que aplicar el cambio de variable 
                                                de $Y = 3X - 2$, lo que es lo mismo, todos sus valores en la nueva variable quedan modificados tal que 
                                                $y_i = 3x_i - 2$. Como sabemos, 
                                                $$CV_X = \frac{\sigma_x}{\bar{x}} \next CV_{aX+b} = \frac{a\sigma_x}{a\bar{x} + b} \next CV_Y = \frac{3\sigma_x}{3\bar{x} - 2}$$
                                                Ahora diviendo entre la media $\bar{x}$ el numerador tenemos
                                                $$CV_Y = \frac{3\sigma_x}{3\bar{x} - 2} = \frac{3CV_X}{3 - \frac{2}{\bar{x}}} $$\\
                                \item Falso. Que las variables sean independientes implica que la covarianza sea nula, pero el recíproco no 
                                                es cierto. Basta considerar el ejemplo siguiente: 
                                                \begin{center}
                                                        \begin{tabular}{c|c c c}
                                                                $X\setminus Y$ & 1 & 2 & 3 \\
                                                                \hline
                                                                -1 & 0 & 1 & 0 \\
                                                                
                                                                0 & 1 & 0 & 1 \\
                                                                
                                                                1 & 0 & 1 & 0 \\
                                                        \end{tabular}
                                                \end{center}
                                                donde podemos ver que la covarianza es nula y sin embargo las variables no son independientes.\\
                                \item Falso. Si son independientes entonces se tiene $m_{rs} = m_{r0} m_{0s} \neq 0$\\
                                \item Falso. El enunciado correcto sería el siguiente: ...los residuos tienen media 0 y la media de los valores
                                                ajustados coincide con la de los observados (es decir, tiene media $\bar{y}$).\\
                                \item Verdadero. Si dos variables son independientes, entonces se cumple que $f_{ij} = f_{i.} f_{.j}, \thinspace \forall i, j$.
                                                Por tanto, aplicando el teorema de Konig se tiene:
                                                $$\sigma_{xy} = \sum_{i=1}^k \sum_{j=1}^p f_{ij} x_i y_j - \bar{x} \bar{y} = 
                                                \sum_{i=1}^k \sum_{j=1}^p f_{i.} f_{.j} x_i y_j - \bar{x} \bar{y} = $$
                                                $$= \sum_{i=1}^k f_{i.} x_i \sum_{j=1}^p f_{.j} y_j - \bar{x} \bar{y} = \bar{x} \bar{y} - \bar{x} \bar{y} = 0$$\\
                                \item Falso. Como las variables son independientes, entonces se tiene que la covarianza es nula, y por tanto el coeficiente
                                                de determinación lineal viene dado por 
                                                $$R^2_{xy} = \frac{\sigma_{xy}^2}{\sigma_x^2 \sigma_y^2} = 0$$
                                                y por tanto, dado que la razón de correlación es mayor o igual que cero, no podemos afirmar que sea estrictamente 
                                                mayor que cero.\\
                                \item Falso. Si consideramos la recta de regresión $Y/X$, los residuos vienen dados por
                                                $$r_{ij} = y_j - \left( \bar{y} + \frac{\sigma_{xy}}{\sigma_x ^2} (x_i - \bar{x})\right)$$
                                                Entonces, suponiendo que la media de los residuos lineales es nula, notando por $n=pk$, luego tenemos 
                                                (sustituyendo el resultado anterior)
                                                $$\sum_{i=1}^k \sum_{j=1}^p \frac{1}{n}r_{ij} = 0 \Leftrightarrow k \sum_{j=1}^p y_j = kp \bar{y}
                                                \leftrightarrow \sum_{j=1}^p y_j = p \frac{1}{n} \sum_{j=1}^p y_j \Leftrightarrow p=n \Leftrightarrow k=1$$
                                                y por tanto, si $k=1$ nos encontraríamos en una unidimensional, lo cual contradice con la hipótesis inicial.\\
                                \item Verdadero. Por ser $ X $ e $Y$ independientes, tenemos que la covarianza es nula $\sigma_{xy}^2  = 0$. Sabemos que la varianza de los residuos 
                                                lineales viene explicada de la siguiente manera:
                                                $$\sigma_y^2 = \sigma_{ey}^2 + \sigma_{ry}^2, \quad \quad \quad \sigma_{ey}^2 = \frac{\sigma_{xy}^2}{\sigma_x^2} = 0$$
                                                $$\next \sigma_y^2 = \sigma_{ry}^2$$
                                \item Verdadero. la varianza sabemos que viene dada por 
                                                $$\sigma_x^2 = \sum_{i=1}^k f_i(x_i - \bar{x})$$
                                                Por definición, la media pertenece al intervalo comprendido entre el valor máximo y el mínimo de todos los $x_i, \thinspace i=1,...,k$.
                                                por lo que tenemos
                                                $$\min(x_i) \leq \bar{x} \leq \max(x_i)$$
                                                Y de la misma manera, podemos deducir que 
                                                $$\min(x_i - \bar{x})^2 \leq \sigma_x^2 \leq \max(x_i - \bar{x})^2$$
                                                Por hipótesis sabemos que la amplitud del intervalo $(\min(x_i), \max(x_i))$ vale 2, luego
                                                $$\max(x_i - \bar{x}) \leq 2 \next \max(x_i - \bar{x})^2 \leq 4$$
                                                $$\min(x_i - \bar{x})^2 \leq \sigma_x^2 \leq \max(x_i - \bar{x})^2 \leq 4 \next \sigma_x^2 \leq 4 $$\\
                                \item Falso. Sabemos que la razón de correlación viene dada por
                                                $$\eta_{x/y} ^2 = 1 - \frac{ECM(X/Y)}{\sigma_x^2}$$
                                                Sustituyendo los dos valores que nos da el enunciado tenemos que $\eta_{x/y} ^2 = 0.25$, en contradicción con que 
                                                valga estrictamente menos que 0.25.
                                \item Verdadero. La media tiene varianza 0 con respecto a sí misma.
                                \item Falso. El coeficiente de correlación lineal viene dado por $R_{xy} = \frac{\sigma_{xy}}{\sigma_x \sigma_y}$. Sabemos que las 
                                                desviaciones típicas se ven afectadas por el valor absoluto del cambio de escala, mientras que la covarianza 
                                                se ve afectada por el producto de ambos cambios de escala. Siendo esto así, tenemos
                                                $$R_{ax+c, by+d} = R_{ax, by} = \frac{\sigma_{ax, by}}{\sigma_{ax} \sigma_{by}} = \frac{ab\sigma_{xy}}{|a||b|\sigma_{x} \sigma_{y}}$$
                                                Como sabemos que $a=-3$ y $b=3$ entonces concluimos
                                                $$R_{x'y'} = \frac{-9\sigma_{xy}}{9 \sigma_{x} \sigma_{y}} = -R_{xy}$$
                                \item Verdadero. Al ser independientes, tenemos $m_{11} = m_{01} m_{10} = \bar{x} \bar{y}$. Despejando ahora $m_{01} = \bar{y}$ tenemos
                                                $\bar{y} = \frac{m_{11}}{m_{10}} = \frac{6}{-3} = -2$. Al aplicar una transformación lineal a la variable $Y$, la media 
                                                se ve afectada tanto por la traslación como por los cambios de escala de la misma manera. Por ello, $\bar{y}' = -3 \bar{y} + 1 = 7$.
                                \item Verdadero. Supongamos la primera recta la de regresión $Y/X$ (y la segunda $ x = -\frac{1}{5}y +\frac{2}{5}$ la de $X/Y$). Tenemos entonces 
                                                $$ -4 = \frac{\sigma_{xy}}{\sigma_x^2} \quad \quad 1 = \bar{y} - \frac{\sigma_{xy}}{\sigma_x^2} \bar{x} = \bar{y} + 4 \bar{x}$$
                                                $$ -\frac{1}{5} = \frac{\sigma_{xy}}{\sigma_y^2} \quad \quad \frac{2}{5} = \bar{x} - \frac{\sigma_{xy}}{\sigma_y^2} \bar{y} = \bar{x} + \frac{1}{5} \bar{y} $$
                                                Tenemos entonces el siguiente sistema:
                                                $$1 = \bar{y} + 4 \bar{x} \quad \quad 2 = 5\bar{x} + \bar{y}$$
                                                Que se reuselve fácilmente y podemos ver que verdaderamente $\bar{x} = 1$. En caso de que dicho valor lo tomara la variable $Y$, 
                                                como el enunciado no proporciona ninguna especificación acerca de las rectas, bastaría con cambiar las variables de nuestro sistema
                                                y obtendríamos el resultado buscado.
                        \end{enumerate}
        

                \newpage
        
                \subsection{Cuestiones Teóricas}         

                \item \textbf{Sea $(X, Y)$ una varaible estadística bidimensional con valores $(x_i, y_j)$ y frecuencias relativas 
                        $f_{ij}$, $i = 1,..., k; \thinspace j = 1,...,p$.}
                
                        \begin{enumerate}
                                \item Dado $j = 1,..., p$, expresa las frecuencias relativas de la distribución condicionada de $X$ a $Y = y_j$,
                                        en términos de las conjuntas, $f_{ij}$, y especifica la media y la varianza de dicha distribución 
                                        condicionada en términos de las frecuencias relativas correspondientes.
                                \item Determina de forma justificada la curva de regresión mínimo cuadrática de $Y$ sobre $X$. Calcula la 
                                        media de los valores ajustados por esta curva.
                                \item Se han observado las variables $(X, Y)$ y se ha obtenido que la recta de regresión de $Y$ sobre $X$ es
                                        $y = -2x + 4$, con $R^2_{xy} = 4/9$, $\bar{x} = 3$ y $\sigma_y = 12$. Determina los valores de $\bar{y}$, 
                                        $\sigma_x$ y la covarianza $\sigma_{xy}$.
                                \item Deducir de forma razonada cómo afecta el cambio de origen y escala al coeficiente de variación $X$.
                                \item Deducir de forma razonada las ecuaciones normales de la recta de regresión mínimo cuadrática de $Y/X$
                                        y obtener la expresión de la recta a partir de dichas ecuaciones. 
                                \item En la regresión lineal de $Y$ sobre $X$, deducir de forma razonada la expresión de la varianza de los 
                                        valores observados en función de los valores ajustados y de los momentos de los residuos. 
                                \item Definir las frecuencias relativas, la media y la varianza de la distribución $Y$ condicionada a $X = x_i$.
                        \end{enumerate}

        
                \newpage

                \subsubsection{Soluciones}

                        \begin{enumerate}
                                \item Frecuencia relativa de $x_i$ en los individuos tales que $Y = y_j$:
                                        $$f_{i/j} = \frac{n_{ij}}{n_{.j}} = = \frac{f_{ij}}{f_{.j}}$$ 
                                        $$\bar{x}_j = \sum_{j=1}^p f_{i/j} x_i \quad \quad \quad \sigma_{x,j}^2 = \sum_{j=1}^p f_{i/j}(x_i - \bar{x}_j  )^2$$\\
                                \item Se deja este ejercicio propuesto para el lector.\\
                                \item Nuestra recta viene dada por $y = ax + b = -2x + 4$. Basándonos en los resultados del apartado anterior, y sustituyendo los valores 
                                        dados por el enunciado, obtenemos
                                        $$
                                        \left.
                                        \begin{aligned}
                                                a = \frac{\sigma_{xy}}{\sigma_x^2} \\
                                                b = \bar{y} - \frac{\sigma_{xy}}{\sigma_x^2} \bar{x}
                                        \end{aligned} \right\}
                                        \next \left.
                                        \begin{aligned}
                                                -2 = \frac{\sigma_{xy}}{\sigma_x^2} \\
                                                4 = \bar{y} - 3 \frac{\sigma_{xy}}{\sigma_x^2}
                                        \end{aligned} \right\}
                                        \next 4 = \bar{y} - 3 (-2)  
                                        $$
                                        $$
                                        \next \bar{y} = - 2
                                        $$
                                        Ya tenemos nuestra primera ecuación. Aplciando ahora que 
                                        $$R^2_{xy} = \frac{\sigma_{xy}^2}{\sigma_x^2 \sigma_y^2} = \frac{4}{9}$$
                                        y junto con la primera ecuación del sistema anterior (y que $\sigma_y = 12$) entonces tenemos:
                                        $$\frac{4}{9} = \frac{\sigma_{xy}^2}{\sigma_x^2 \sigma_y^2} = \sigma_{xy}\frac{\sigma_{xy}}{\sigma_x^2} \frac{1}{\sigma_y^2} = 
                                        \sigma_{xy}(-2)\frac{1}{144} \next \sigma_{xy} = -32
                                        $$
                                        Y retomando ahora nuestra primera ecuación, obtenemos de manera directa que $\sigma_x = 4 $.\\
                                \item Veamos cómo es afectado por cambios de origen, pero sin embargo no afectado por cambios de escala.
                                        $$CV_X = \frac{\sigma_x}{\bar{x}}  \next  CV_{aX+b} = \frac{\sigma_{ax+b}}{\overline{{ax+b}}}$$
                                        Como bien sabemos, la media se ve afectada tanto por cambios de escala como por cambios de origen, 
                                        al igual que la varianza por cambios de escala de la siguiente manera:
                                        $$CV_{aX+b} = \frac{\sigma_{ax+b}}{\overline{{ax+b}}} = \frac{|a|\sigma_{x}}{a\overline{x} + b}$$
                                        Donde vemos que en caso de que $b=0$ (sin cambio de origen, sólo de escala) resulta que 
                                        $CV_{aX} = CV_{X}$, pero dicha igualdad no se cumple para cualquier otro valor de $b$, luego concluimos
                                        que el coeficiente de variación es invarainte ante cambios de escala, pero sí es sensible a cambios de origen.\\
                                \item Queremos que $Y/X$ sea aproximado por una distribución de la forma $y=ax+b$ tal que la función
                                        $$\lambda(a,b) = \sum_{i=1}^k \sum_{j=1}^p f_{ij} [y_j - (ax_i + b)]^2$$
                                        sea mínima (notemos que es la varianza donde hemos sustituido la recta que buscamos en la media de $y$). Para calcular 
                                        que sea mínima, calculamos las parciales con respecto a las variables $a$ y $b$ e igualamos a cero (para obtener los 
                                        posibles extremos relativos) de donde obtenemos las siguientes ecuaciones normales:
                                        $$
                                        \left.
                                        \begin{aligned}
                                                &\frac{\partial \lambda(a,b)}{\partial a} = 0 \\
                                                &\frac{\partial \lambda(a,b)}{\partial a} = 0 \\                                                
                                        \end{aligned} \right\}
                                        \next \left.
                                        \begin{aligned}
                                                m_{11} = am_{20} + bm_{10}\\
                                                m_{01} = am_{10} + b\\
                                        \end{aligned} \right\}
                                        \next \left.
                                        \begin{aligned}
                                                a = \frac{\sigma_{xy}}{\sigma_x^2} \\
                                                b = \bar{y} - \frac{\sigma_{xy}}{\sigma_x^2} \bar{x}
                                        \end{aligned} \right\}
                                        $$
                                        de donde hemos obtenido los elementos de nuestra recta $y=ax+b$ que buscábamos. Notemos que la varianza no puede ser nula, 
                                        ya que en tal caso tendremos que $X$ es constante, y por tanto no tendría sentido calcular la recta de regresión (los datos 
                                        ya dibujarían una recta).\\
                                \item Calculemos primero la varianza de los valores ajustados ($\sigma^2_{ey}$) y luego la de los residuos ($\sigma^2_{ry}$):
                                        $$
                                        \sigma^2_{ey} = \sum_{i=1}^k f_{i.} \left(\bar{y} + \frac{\sigma_{xy}}{\sigma_x^2} (x_i - \bar{x}) - \bar{y} \right)^2 = 
                                        \sum_{i=1}^k f_{i.} \left(\frac{\sigma_{xy}}{\sigma_x^2} (x_i - \bar{x}) \right)^2 =
                                        $$

                                        $$
                                        = \frac{\sigma_{xy}^2}{\sigma_x^4} \sum_{i=1}^k f_{i.} \left( x_i - \bar{x} \right)^2 = \frac{\sigma_{xy}^2}{\sigma_x^2}
                                        $$

                                        $$
                                        \sigma^2_{ry} = \sum_{i=1}^k \sum_{j=1}^p f_{ij} \left(y_j - \bar{y} - \frac{\sigma_{xy}}{\sigma_x^2} (x_i - \bar{x}) \right)^2 =
                                        $$

                                        $$
                                        = \sum_{i=1}^k \sum_{j=1}^p f_{ij} (y_j - \bar{y})^2 + \frac{\sigma_{xy}^2}{\sigma_x^4} \sum_{i=1}^k \sum_{j=1}^p f_{ij}(x_j - \bar{x})^2
                                        - 2\sum_{i=1}^k \sum_{j=1}^p f_{ij} (y_j - \bar{y})\frac{\sigma_{xy}}{\sigma_x^2}(x_j - \bar{x})
                                        $$

                                        $$
                                        = \sigma_y^2 + \frac{\sigma_{xy}^2}{\sigma_x^4} \sigma_x^2 - 2\frac{\sigma_{xy}}{\sigma_x^2}\sigma_{xy} = 
                                        \sigma_y^2 + \frac{\sigma_{xy}^2}{\sigma_x^2} -2\frac{\sigma_{xy}^2}{\sigma_x^2} = \sigma_y^2 - \frac{\sigma_{xy}^2}{\sigma_x^2}
                                        $$

                                        Y por tanto tenemos que la suma de ambos resulta:
                                        $$
                                        \sigma^2_{ey} + \sigma^2_{ry} = \frac{\sigma_{xy}^2}{\sigma_x^2} + \sigma_y^2 - \frac{\sigma_{xy}^2}{\sigma_x^2} \next \sigma_y^2 = \sigma^2_{ey} + \sigma^2_{ry}
                                        $$\\
                                \item Mismo proceso que el apartado $a)$, cambiando subíndices.
                        \end{enumerate}

        \end{enumerate}       

        \newpage

        
        \section{Probabilidad como Conjuntos}

        \begin{enumerate}
                \subsection{Verdadero/Falso}
                \item \textbf{Justificar la veracidad o falsedad de las siguientes afirmaciones:}

                        \begin{enumerate}
                                \item Si A y B están contenidos en C, entonces $P(\bar{C}) < P(\bar{A}) + P(\bar{B})$
                                \item $P(B_i/A) = \frac{P(A / B_i)P(B_i)}{P(A)}$
                                \item Si A y B son dos sucesos independientes, entonces $P(A\cup B) = P(A) + P(B)$
                                \item Si $\{B_i\}_{i=1,2,3,...}$ son sucesos incompatibles, exhaustos y de probabilidades no nulas, entonces para un suceso
                                        arbitrario A se tiene:
                                        $$P(A / B_j) = \frac{P(B_j / A) P(A)}{\sum_{i=1}^n P(B_i / A)P(A)}$$
                                \item $P(A \cap B) \geq P(A) + P(B) - 1$
                                \item $P(A\cap B \cap C) = P(B/A\cap C) P(A/C) P(C)$
                                \item $P(A \cap \bar{B}) = P(A)P(\bar{B})$
                                \item Si A y B son sucesos independientes, entonces $\bar{A}$ y $\bar{B}$ también lo son.
                                \item Si $A$ y $\bar{B}$ son independientes, entonces $\bar{A}$ y $B$ también lo son.
                                \item Si $\bar{A}$ y $B$ son independientes, entonces $P(A\cup B) = P(\bar{A})P(B) + P(A)$
                                \item $P(A \cap \bar{B}) = P(A) - P(A \cap B)$
                                \item $P(A \cap \bar{B}) = P(A) - P(B)$
                                \item $P(A) + P(B) > P(A\cap B)$ implica que $A$ y $B$ son sucesos incompatibles.
                                \item $P((A-B)/A) = 1 - P(B)$ implica que $A$ y $\bar{B}$ son sucesos independientes.
                                \item $P(\bigcup_{i=1}^n A_i) \geq 1 - \sum_{i=1} ^n P(A_i)$
                                \item $P(A) = \sum_{i=1}^n P(A/B_i)P(B_i)$
                                \item $P(A/B) = P(A\cap B) / P(A)$
                                \item Si A, B y C son tres sucesos independientes dos a dos, entonces \\ $P(A/B) = P(B/A)$.
                                \item Si A, B y C son tres sucesos independientes dos a dos, entonces \\ $P(A\cap B \cap C) = P(A)P(B)P(C)$.
                                \item Si A, B y C son tres sucesos independientes dos a dos, entonces \\ $P(A/(B\cap C)) = P(A)$.
                                \item Si A, B y C son tres sucesos independientes dos a dos, con C independiente de $A\cup B$ entonces
                                        $P((\bar{A}\cup \bar{B})/C) = 1 - P(A)P(B)$.
                                \item Si A y B son dos sucesos incompatibles e independientes, entonces \\ $P(A \cup B) \geq 1 - P(\bar{A}) P(\bar{B})$
                                \item Si A y B son dos sucesos incompatibles e independientes, entonces \\ $P(A) > 0 \next P(\bar{B}/A)<1$ 
                        \end{enumerate}


                \newpage

                \subsubsection{Soluciones}

                        \begin{enumerate}
                                \item Verdadero. Como $A\subset C$, tenemos que $P(A)\leq P(C)$, y entonces tenemos que 
                                        $P(\bar{A}) \geq P(\bar{C})$ (se realiza el mismo proceso con $B$, obteniendo $P(\bar{B}) \geq P(\bar{C})$).
                                        Sumando ambas desigualdades tenemos $P(\bar{A}) + P(\bar{B}) \geq 2P(\bar{C})$ y por tanto concluimos
                                        $P(\bar{A}) + P(\bar{B}) > P(\bar{C})$.\\
                                \item Verdadero. Basta con ver que $P(A\cap B)=P(A/B)P(B)$, y vemos que el enunciado cumple la propiedad de la probabilidad
                                        condicionada $P(B / A)=\frac{P(A\cap B)}{P(A)}$.\\
                                \item Falso. Dicha igualdad sólo se cumple cuando alguna de los dos conjuntos tenga probabilidad nula, para que se
                                        cumpla la condición de independencia $ P(A\cap B) = P(A)P(B)$. Con que una de as dos probabilidades sea no nula, no 
                                        se cumple la igualdad dada.\\
                                \item Falso. La definición correcta del Teorema de Bayes concluiría con la siguiente igualdad:
                                        $$P(A_j / B) = \frac{P(B / A_j) P(A_j)}{\sum_{i=1}^n P(B / A_i)P(A_i)}$$\\
                                \item Verdadero. $P(A \cap B) \geq P(A) + P(B) - 1 \next  1 \geq P(A) + P(B) - P(A \cap B) = P(A \cup B) \next 1 \geq P(A \cup B)$\\
                                \item Verdadero. $$P(B/A\cap C)P(A/C)P(C) = \frac{P(A\cap B\cap C)}{P(A\cap C)}\frac{P(A\cap C)}{P(C)} P(C) = $$
                                        $$P(A\cap B\cap C)$$\\
                                \item Falso. Esto sólo se cumple cuando los sucesos son independientes. Si no se da el caso, el enunciado es erróneo.\\
                                \item Verdadero. $$P(\bar{A})P(\bar{B}) = (1-P(A))(1-P(B))=1-P(A)-P(B) + P(A)P(B)=$$
                                        $$=1-P(A)-P(B) + P(A\cap B) = 1 - P(A \cup B) = \bar{P(A \cup B)} = P(\bar{A} \cap \bar{B})$$\\
                                \item Verdadero. Demostración análoga a la anterior.\\
                                \item Verdadero. Sabemos que $P(A \cup B) = P(A) + P(B)-P(A \cap B)$, luego la demostración se basa en probar que 
                                        $P(B)-P(A \cap B) = P(\bar{A})P(B)$, y esto es tan fácil como ver que $P(B)-P(A \cap B) = P(\bar{A} \cap B)$
                                        y que $\bar{A}$ y $B$ son independientes.\\
                                \item Verdadero. Definición de probabilidad de intersección.\\
                                \item Falso. Sólo cierto en caso de que $P(A)=1$.\\
                                \item Falso. Para $P(A)=0.8, \thinspace P(B)=0.7, \thinspace P(A \cap B)= 0.6$ se cumple la desigualdad y no son incompatibles
                                        (intersección no nula).\\
                                \item Verdadero. Llegamos a la independencia entre A y B: 
                                        $$P((A-B)/A) = P((A\cap \bar{B})/A) = \frac{P(A\cap \bar{B})}{P(A)} = 1-\frac{P(A\cap B)}{P(A)}$$
                                        $$\next \frac{P(A\cap B)}{P(A)} = P(B) \next P(A\cap B)=P(A)P(B)$$
                                        Y ahora como $A$ y $B$ son independientes, $A$ y $\bar{B}$ también lo son.\\
                                \item Falso. Supongamos $P(A)=0.1, \thinspace P(B)=0.2, \thinspace P(A \cap B)= \emptyset$. Es fácil ver que no se cumple la igualdad.\\
                                \item Falso. Si la intersección de dos conjuntos $B_i$ es no nula, la igualdad dada no se cumple (Teorema de la Probabilidad Total).\\
                                \item Falso. El denominador debería ser $P(B)$ por definición de probabilidad condicionada.\\
                                \item Falso. Para que se de la igualdad ha de cumplirse que $P(A) = P(B)$.\\
                                \item Falso. Basta considerar Que la intersección de los tres conjuntos sea nula, pero que los sucesos sean no nulos.\\
                                \item Falso. Consideremos igual que en el apartado anterior el caso en que $P(A\cap B\cap C) = 0$, y tendríamos
                                $$P(A/(B\cap C)) = \frac{P(A\cap B\cap C)}{P(B\cap C)} = 0$$
                                En cuyo caso sólo se cumpliría en el caso $P(A)=0$, pero sería falso en cualquier otro caso.\\
                                \item Verdadero. $$P((\bar{A}\cup \bar{B})/C) = P((\bar{{A}\cap {B}})/C) = 1-P(({A}\cap {B})/C) = 1- \frac{P(A\cap B\cap C)}{P(C)} = $$
                                        $$=1- \frac{P(A)P(B)P(C)}{P(C)} = 1- P(A)P(B)$$\\
                                \item Verdadero. Si son incompatibles, la intersección es nula, y por tanto de la independencia se tiene que $A$ o $B$ al menos uno de
                                        los dos es un suceso nulo. Si sólo uno es nulo (supongamos $A$), tendríamos $P(A\cup B) = P(B)$ y $P(\bar{A})=1$, y por tanto se 
                                        daría la igualdad en la desigualdad dada. En caso de que ambos fueran nulos, entonces la unión es nula, y por tanto se daría de nuevo
                                        la igualdad (ya que $P(\bar{A})=P(\bar{B})=1$ y por ello $1-P(\bar{A})P(\bar{B})=0$).\\
                                \item Falso. Si $A$ es no nulo, por lo mencionado en el apartado anterior, tenemos que $P(B)=0$ y así $P(\bar{B}) = 1$ lo que lo convierte
                                        en suceso seguro. Por ello al calcular
                                        $P(\bar{B}/A)=\frac{P(\bar{B}\cap A)}{P(A)} = \frac{P(A)}{P(A)} = 1$. Luego no se cumple la desigualdad dada.
                        \end{enumerate} 


                \newpage

                \subsection{Cuestiones Teóricas}

                        \begin{enumerate}
                                \item Si A y B son sucesos independientes, $\bar{A}$ y $\bar{B}$ también lo son.\\
                                \item Definir la función de probabilidad condicionada a un suceso no nulo y demostrar 
                                        que satisface los axiomas de Kolmogorov.\\
                                \item Especificar las condiciones para que tres sucesos sean independientes.\\
                                \item Demostrar que $1 - P(\bar{A}) - P(\bar{B}) \leq P(A \cap B) \leq P(A) + P(B)$\\
                                \item Sean $\{A_n\}_{n\in \N} \in \mathcal{A}$ sucesos no nulos, incompatibles dos a dos, cuya unión es $\Omega$ y sea 
                                        $B \in \mathcal{A}$ un suceso arbitrario. Suponiendo que las probabilidades $P(A_n), \thinspace n \in \N$, son conocidas,
                                        y que también lo son las probabilidades condicionadas $P(B/A_n), \thinspace n \in \N$, deducir la expresión para calcular 
                                        $P(B)$ a partir de ellas.
                        \end{enumerate}

                
                \newpage

                \subsubsection{Soluciones}
                        
                        \begin{enumerate}
                                \item Solución dada en el apartado $h)$ de la sección anterior.
                                \item Dado un espacio de probabilidad $(\Omega, \mathcal{A}, P)$, y un suceso no nulo $A\in \mathcal{A}$, la aplicación
                                        $$
                                        \begin{aligned}
                                                P(\cdot/A) : \thinspace &\mathcal{A} \rightarrow \R \\
                                                                        &B \mapsto P(B/A)
                                        \end{aligned}
                                        $$
                                        es una función de probabilidad sobre $(\Omega. \mathcal{A}) $ cumpliendo:
                                        \begin{enumerate}
                                                \item $P(B/A) \geq 0, \thinspace \forall B \in \mathcal{A}$ \\
                                                        Dado que $A$ es un suceso no nulo, y que $P(A\cap B) \leq P(A)$, tenemos que la probabilidad condicionada
                                                        está bien definida.
                                                \item $P(\Omega/A)=1$.\\
                                                        $P(\Omega/A) = \frac{P(\Omega \cap A)}{P(A)} = \frac{P(A)}{P(A)} = 1$
                                                \item Tercer axioma de Kolmogorov, se deja demostración al lector como ejercicio.
                                        \end{enumerate}
                                \item Para que tres sucesos cualesquiera sean independientes, ha de cumplirse
                                        $$P(A\cap B \cap C)=P(A)P(B)P(C)$$
                                \item La fórmula para el cálculo de $P(B)$ viene dado por 
                                        $$P(B) = \sum_{n=1}^{+\infty} P(B/A_n)P(A_n)$$
                        \end{enumerate}
        
        \end{enumerate}
        
        \newpage

        \section{Variables Aleatorias y Distribuciones Discretas}

        \begin{enumerate}

                \subsection{Opción Múltiple}
                \item \textbf{Responde correctamente a las siguientes cuestiones:}
                
                        \begin{enumerate}
                                \item \textbf{Sea $X$ una variable aleatoria discreta y considérese la nueva v.a. $Y = g(X)$, 
                                        donde $g$ es una función continua estrictamente decreciente. Si $h = g^{-1}$, entonces 
                                        la función de distribución de $Y$ es:} 
                                        \begin{enumerate}
                                                \item $F_Y (y) = 1 - F_X (h(y))$
                                                \item $F_Y (y) = F_X (h(y))$
                                                \item $F_Y (y) = 1 - F_X (h(y)) + P(Y=y)$
                                                \item $F_Y (y) = [1 - F_X (h(y))][h'(y)]$
                                        \end{enumerate}
                                
                                \item \textbf{Sea $X$ una v.a. discreta con media $m$ y desviación típica 0. Entonces se puede afirmar que:}
                                        \begin{enumerate}
                                                \item $P(X \neq 0) = 0$
                                                \item $P(X = 0) = 1$
                                                \item $P(X \neq 0) = 1$
                                                \item $X=m$
                                        \end{enumerate}

                                \item \textbf{Sea $X$ una variable aleatoria continua y considérese la nueva v.a. $Y = aX + b$. Entonces:}
                                        \begin{enumerate}
                                                \item $F_Y (y) = 1 - F_X \left(\frac{y-b}{a}\right)$
                                                \item $F_Y (y) = \left[1 - F_X \left(\frac{y-b}{a}\right)\right] |a|^{-1}$
                                                \item $f_Y (y) = \frac{1}{|a|} f_X \left(\frac{y-b}{a}\right) $
                                                \item $f_Y (y) = |a| f_X (ax+b)$
                                        \end{enumerate}
                                
                                \item \textbf{Sea $X$ una v.a. con función de distribución $F_X(x) = 2x, \s 0<x<1/8, \ss F_X(x) = 1/2, \s 
                                        1/8 \leq x < 1/2 $ y $F_X (x) = \frac{x+1}{2}, \s 1/2 \leq x < 1$. Entonces se tiene: }
                                        \begin{enumerate}
                                                \item El único valor de la mediana es 1/2.
                                                \item $P\left(X < \frac{1}{8}\right) + P\left(X > \frac{1}{8}\right) = \frac{3}{4}$.
                                                \item $P\left(\frac{1}{8} \leq X < \frac{1}{2}\right) > P\left(X = \frac{1}{8}\right)$.
                                                \item $P\left(X \geq \frac{1}{8}\right) > P\left(X < \frac{1}{2}\right)$.
                                        \end{enumerate}

                        \newpage
                                \item \textbf{Sea $X$ una variable aleatoria simétrica tal que el primer cuartil es 1.5, el rango intercuartílico
                                        es 1 y el coeficiente de variación 1/4. Entonces:}
                                        \begin{enumerate}
                                                \item Todos los puntos del intervalo $[1.75, \s 2.25]$ son mediana.
                                                \item $P(-2 < X < 4) = P(0 < X < 6)$.
                                                \item El rango intercuartílico de $X$ duplica al de la variable tipificada.
                                                \item $P \left(-1 < \frac{2X - 4}{3} < 1 \right) \geq 0.9$.
                                        \end{enumerate}       
                                
                                \item \textbf{Sea $(\Omega, \mathcal{A}, P)$ un espacio de probabilidad arbitrario y $X: \Omega \rightarrow \R$. Entonces:}
                                        \begin{enumerate}
                                                \item Si $X$ es una v.a., su distribución de probabilidad satisface 
                                                        $P_X((a,b]) = 1 - P_X((b, +\infty)) - P_X((-\infty, a))$.
                                                \item $X$ es v.a. $\Leftrightarrow P(X^{-1}(B)) \in \mathcal{A}, \ss \forall B \in \mathcal{B}$.
                                                \item Si $X$ es variable aleatoria y $F_X$ es su función de distribución, entonces 
                                                        $P(a\leq X < b) = F_X(b^{-})-F_X(a^{-})$.
                                                \item Si $\Omega = [-1,1]$ y $\mathcal{A} = \{\emptyset, \Omega, [-1, 0), [0, 1]\}$, la función 
                                                $X(\omega) = \omega, \ss \forall \omega \in \Omega$, es variable aleatoria sobre $(\Omega, \mathcal{A}, P)$.
                                        \end{enumerate}
                        \end{enumerate}

                
                \newpage

                \subsubsection{Soluciones}
                        
                        \begin{enumerate}
                                \item Opción 1). \textit{Si g fuera monótona creciente...}\\
                                        Al ser $g$ una función continua y estrictamente monótona, tenemos que dicha función es biyectiva en el dominio
                                        en el que esté definida. Por definición sabemos que 
                                        $$F_Y(y) = P(Y \leq y) = P(X\leq g^{-1}(y)) = F_X (g^{-1}(y))$$
                                        Por ello, llamando ahora $g^{-1}=h$ tenemos $F_Y(y) = F_X (h(y))$.\\ 
                                        *\textit{Como g es monótona decreciente...}\\
                                        Basta considerar ahora que, por ser decreciente, si el intervalo maximal de definición viene dado por $[a, b]$, su imagen 
                                        por $g$ viene dada por $[g(b), g(a)]$, con $a\leq b$. Por ello, tenemos
                                        $$F_Y(y) = P(Y \leq y) = P(X\geq g^{-1}(y)) = 1 - P(X\leq g^{-1}(y)) =$$
                                        $$= 1 - F_X (g^{-1}(y))$$
                                        Por ello, llamando ahora $g^{-1}=h$ tenemos $F_Y(y) = 1 - F_X (h(y))$.\\
                                \item Opción 4). Si la desviación típica vale 0 significa que el intervalo que contiene a todos los elementos es de longitud nula, 
                                        por ello sabemos que todos los elementos del conjunto han de ser el mismo. Al tener media $m$, y saber que todos
                                        los elementos de la v.a. son el mismo (es decir, son un único valor), concluimos que $X=m$. Los dos primeros 
                                        apartados es fácil desmentirlos considerando cualquier $m \neq 0$, y para el tercero $m=0$ conduce también a
                                        un enunciado erróneo. \\
                                \item Opción 3). Vamos a suponer $a,b \neq 0$, ya que en dicho caso el resultado sería trivial, y no hay ninguna opción para ello. Podemos
                                        ver que al tratarse de una transformación lineal sabemos que es derivable, y como la derivada viene dada por el valor de $a$,
                                        sabemos que la transformación $g(X) = aX + b$ es estrictamente monótona. Por el apartado $a)$ sabemos que $F_Y(y) = F_X (g^{-1}(y))$,
                                        y por tanto calculando la inversa de la transformación y sustituyendo obtenemos $g^{-1}(y) = \frac{y-b}{a}$ y así
                                        $F_Y(y) = F_X (\frac{y-b}{a})$. Vemos que esta igualdad no la dispone ninguna de las opciones del enunciado, luego 
                                        procedamos como sigue. Para ver su función de distribución basta con aplicar el teorema del cambio de variable para
                                        transformaciones de continua a continua, dado por 
                                        $$ f_Y(y) = f_X(g^{-1}(y)) \left| \frac{d g^{-1}(y)}{dy} \right| \next 
                                        \left| \frac{d g^{-1}(y)}{dy} \right| = \left| \frac{d \frac{y-b}{a}}{dy} \right| = \frac{1}{|a|}$$
                                        $$ \next f_Y(y) = \frac{1}{|a|}f_X \left(\frac{y-b}{a}\right)$$
                                \item Opción 2). Tengamos en cuenta que la función de distribución tiene una discontinuidad de salto en $x=1/8$, luego dicho valor, a pesar 
                                        de ser distribución continua, toma valor no nulo. En concreto toma probabilidad con valor $P(X=1/8)= F(x^+)-F(x^-) = 1/2 - 1/4 = 1/4$.
                                        Esta distribución presenta otra discontinuidad de salto con la misma probabilidad en $x=1/2$. Vemos por ello que 
                                        $$P\left(X < \frac{1}{8}\right) + P\left(X > \frac{1}{8}\right) = 1 - P\left(X = \frac{1}{8}\right) = 1 - \frac{1}{4} = \frac{3}{4}$$
                                \item Opción 2). Como el primer cuartil vale 1.5 y el rango intercuartílico es 1, nos indica que el segundo cuartil (la mediana) se encuentra en 
                                        $x=2$, luego nos encontramos ante una distribución simétrica centrada en 2, y por ser simétrica, dicho valor coincide con su media.
                                        Como $CV_X = \frac{\sigma_x}{\bar{x}} = \frac{1}{4}$, tenemos que la desviación típica toma valor 1/2, y vemos por ello que 
                                        $P(-2 < X < 4) = P(0 < X < 6)$ ya que todos los datos se encuentran comprendidos en el intervalo $(0,4)$.
                                \item Opción 3). Definición de probabilidad de un intervalo.
                                
                        \end{enumerate}


                \newpage

                \subsection{Verdadero/Falso}
                \item \textbf{Justificar la veracidad o falsedad de las siguientes afirmaciones:}

                        \begin{enumerate}
                                \item La varianza de la distribución binomial es siempre inferior a su esperanza.
                                \item La distribución de Poisson es un caso particular de la binomial cuando n es grande y $p$ pequeño.
                                \item En sucesivos experimentos independientes éxito/fracaso, con probabilidad de éxito $p$ constante,
                                        el número de realizaciones hasta encontrar el r-ésimo éxito sigue una distribución normal negativa.
                                \item La distribución hipergeométrica toma valores entre cero y n con probabilidades no nulas.
                                \item El momento centrado de orden 2 de una variable aleatoria nunca puede ser inferior al momento no centrado
                                        de orden 2 de dicha variable.
                                \item Si $X$ es una v.a. no negativa cuya esperanza existe, entonces se tiene
                                        $P(X\geq \varepsilon) \leq \frac{E[X]}{\varepsilon}, \thinspace \forall \varepsilon >0$.
                                \item Si $E[X] = 2$ y $Var[X] = 3$, entonces $P(-3 < X < 7) \geq 3/5$.
                                \item Una variable aleatoria que es transformada de otra no tiene por qué ser siempre del mismo tipo que ésta.
                                \item La distribución geométrica modela el número de fracasos que han ocurrido antes de que ocurra el primer éxito.
                                \item El momento no centrado de orden 2 de una v.a. nunca puede ser inferior al cuadrado del momento centrado de 
                                        orden 1 de dicha variable.
                                \item Sea $X$ una variable aleatoria continua definida en todo $\R$. Entonces se tiene 
                                        $F_X (x) = 1 - F_{-X} (x), \thinspace \forall x \in \R^+$.
                                \item Si $p=0.4$ en una distribución binomial, el cálculo de $\frac{7!}{3! 4!} (0.4)^3 (0.6)^4$ nos da la posibilidad 
                                        de conseguir exactamente tres éxitos en 7 ensayos.
                                \item La varianza de la distribución de Poisson aumenta cuando aumenta su media.
                                \item La distribución binomial con $n=1$, es la distribución de Bernoulli.
                                \item La distribución geométrica es un caso particular de la hipergeométrica si $r$ (número de éxitos) es igual a 1.
                                \item Si la función generatriz de momentos de $X$ es $M_X(t) = (pe^t + 1 -p)^6, \ss t \in \R$ con $E[X] = 1.8$, 
                                        entonces $p=2$.
                                \item Si $E[X] = 2/3$ y la función generatriz de momentos de $X$ es $M_X(t) = \frac{p}{1-0.4e^t}, \ss t< -\ln 0.4$,
                                        entonces $p=0.6$.
                                \item Si $X$ es v.a. con función de distribución $F_X(x) = x, \ss 0 \leq x \leq 1$, su función generatriz de momentos
                                        vale $M_X(t) = e^t /t, \ss t \in \R$.
                        \end{enumerate}
                

                \newpage 

                \subsubsection{Soluciones}

                        \begin{enumerate}
                                \item Verdadero. $E[X] > Var[X]$, siendo $X$ una distribución binomial, donde $n \in \N$ y $ p \in (0,1)$. 
                                        Tenemos entonces
                                        $$np > np(1-p) \next 1>1-p \next p>0$$\\
                                \item Verdadera. La distribución de Poisson se aproxima a la binomial cuando $n>20$ y $p\leq 0,1$. Se la conoce 
                                        por ello como la ley de los sucesos raros. El enunciado es un poco impreciso pero lo damos como válido.\\
                                \item Falso. Una $X$ con distribución binomial negativa describe el número de fracasos antes de que ocurra el 
                                        r-ésimo éxito.\\
                                \item Verdadero. Por definición, sabemos que si tenemos una muestra aleatoria de tamaño $n$, entonces 
                                        $x=0, ... , n$.\\
                                \item Falso. El enunciado afirma que $m_2 \leq \mu_2$, siendo éstos los momentos de una v.a. $X$. Por definición sabemos
                                        que 
                                        $$\left.
                                        \begin{aligned}
                                                &m_2 = E[X^2] \\
                                                &\mu_2 = E[(X- E[X])^2] = E[X^2]- E[X]^2
                                        \end{aligned}
                                        \right\}
                                        $$
                                        Si suponemos $m_2 \leq \mu_2$ tenemos
                                        $$E[X^2] \leq E[X^2] - E[X]^2  \next  E[X]^2 \leq 0$$
                                        Lo cual es una contradicción.\\
                                \item Verdadero. Nos encontramos ante el enunciado del teorema de Markov de la Desigualdad Básica, cuya demostración
                                        se puede ver detenidamente en el siguiente enlace: \url{http://www.ugr.es/~cdpye/CursoProbabilidad/pdf/P_T04_DesigualdadBasica.pdf}\\
                                \item Verdadero. Aplicando la desigualdad de Chebychev en una de sus variantes dada por 
                                        $$P\left( - k\sqrt{Var[X]} < X - E[X] < k\sqrt{Var[X]}\right) \geq 1- \frac{1}{k^2}, \quad \forall k>0$$
                                        $$P\left( - k\sqrt{3} < X - 2 < k\sqrt{3}\right) = P\left( 2 - k\sqrt{3} < X < 2 + k\sqrt{3}\right)$$
                                        Tomando $k = \frac{5}{\sqrt{3}}$ obtenemos 
                                        $$P\left( - 3 < X < 7\right) \geq \frac{22}{25} \geq \frac{3}{5}$$\\
                                \item Falso. Por resultados vistos anteriormente hemos visto que, siendo $g$ una transformación medible, dicha transformación 
                                        conserva todas las propiedades de la variable primaria.\\
                                \item Verdadero. Definición de la distribución geométrica.\\
                                \item Verdadero. El enunciado afirma que $m_2 \geq \mu_1^2$. Aplicando lo visto en el apartado $e)$ tenemos $ E[X^2] \geq E[X- E[X]] = 0$.
                                        Como $E[X^2] = \sum_i \frac{x_i^2}{n}$, al ser suma de cuadrados vemos claramente que es suma de elementos no negativos,
                                        y por tanto afirmamos que $E[X^2] \geq 0$.\\
                                \item Falso. Sabemos por definición que $F_X (x) = P(X\leq x) $, donde tenemos por hipótesis que $x \in \R^+$. Consideremos la transformación lineal 
                                        (y por tanto diferenciable) dado por $Y = g(X) = -X$, de donde sabemos que 
                                        $$F_Y(y) = F_X(g^{-1}(y)) = F_X(-y)$$ 
                                        Como $Y=-X$ deshaciendo los cambios obtenemos 
                                        $$F_{-X}(-x) = F_X(x)$$
                                        Para el último paso de la prueba necesitaríamos la condición $F_{-X}(-x) = 1 - F_{-X}(x)$, lo cual sólo se daría en caso de que la distribución
                                        fuera simétrica centrada en 0, lo cual no se supone en la hipótesis. Luego suponiendo cualquier otro tipo de distribución continua se llega al 
                                        contraejemplo.\\
                                \item Verdadero. Extrayendo de la hipótesis los datos de la fórmula del cálculo de probabilidades de la distribución binomial,
                                        $P(X=x) = \left(\begin{matrix} n \\ x \end{matrix}\right) p^x (1-p)^{n-x}$, obtenemos los datos $n=7, \thinspace x=3$, 
                                        luego podemos afirmar que $X$ describe el numero de éxitos en $n$ repeticiones independientes.\\
                                \item Verdadero. Ya que, en una distribución de Poisson la media es igual a su varianza, es fácil ver que el enunciado es cierto.\\
                                \item Verdadero. Si tomamos $n=1$ vemos forzado a que $x$ sólo pueda tomar valores 0 y 1, y en su función de probabilidad vemos que, como 
                                        la binomial viene dada por $P(X=x) = \left(\begin{matrix} n \\ x \end{matrix}\right) p^x (1-p)^{n-x}$, al suponer
                                        $n=1$ entonces tenemos $P(X=x) = p^x (1-p)^{1-x}$ que es precisamente una Bernoulli con probabilidad $p$.\\
                                \item Falso. La distribución geométrica es un caso particular cuando $r=1$, pero no de la hipergeométrica. Se trata de un caso particular
                                        de la binomial negativa.
                                \item Falsa. Sabemos que $E[X] = \left.\frac{dM_X (t)}{dt} \right|_{t=0}$, por ello derivando y evaluando en 0 tenemos 
                                        $E[X] = 6p$. Como por hipótesis sabemos que $E[X] = 1.8$, concluimos que $p \neq 2$.
                                \item Verdadera. Procedimiento similar al anterior.
                                \item Falsa. Si calculamos la media de la distribución vemos que tenemos $E[X]=1/2$, Además, la función generatriz de momentos ha de cumplir
                                        $$E[X] = \left.\frac{dM_X (t)}{dt} \right|_{t=0}$$
                                        Luego, derivando la función generatriz con respecto a $t$ obtenemos 
                                        $$E[X] = \left. e^t \left(\frac{1}{t} - \frac{1}{t^2}\right) \right|_{t=0} = -\infty$$
                                        Y por lo tanto vemos que no coinciden.
                        \end{enumerate}

                
                \newpage
                
                \subsection{Cuestiones Teóricas}

                        Estas cuestiones y demostraciones son resultados vistos en clase. Aún así os las dejo propuestas, así como lo han hecho los profesores con estas 
                        preguntas en los últimos años, para que podáis ir lo mejor preparadxs posible al examen final.
                        \begin{enumerate}
                                \item Probar que la distribución de probabilidad es una función de probabilidad.
                                \item Si $X$ es una variable aleatoria con media 1 y varianza 2, dar una cota inferior para la probabilidad
                                        de que la variable tome valores en $[-4, 11]$.
                                \item Explicar el modelo que da lugar a la distribución binomial negativa y especificar su función masa de probabilidad.
                                \item Deducir la función generatriz de momentos de la distribución geométrica.
                                \item Definir formalmente el concepto de variable aleatoria especificando la distribución de todos los elementos
                                        que aparecen.
                                \item Definir la distribución de probabilidad de una variable aleatoria y probar que satisface los axiomas de Kolmogorov.
                                \item Definir la función de distribución de una variable aleatoria y enumerar las propiedades que caracterizan de manera
                                        genérica a esta función.
                                \item Formular y demostrar la desigualdad de Chebishev.
                                \item Distribución binomial. Definición. Cálculo de la media y la varianza.
                                \item Distribución de Poisson. Definición. Función de distribución y función generatriz de momentos. Cálculo de la media.
                                
                                        
                                
                        \end{enumerate}

\begin{comment}

                \newpage

                \subsubsection{Soluciones}
                                Estas cuestiones se dejarán resueltas de cara a la convocatoria ordinaria, ya que todas estas son definiciones y demostraciones
                        que se encuentran a disposición del alumnado en múltiples plataformas, y que de cara al parcial del próximo viernes no resultan 
                        relevantes. Mucho ánimo y nos vemos este viernes en PRADO.

                        \begin{enumerate}
                                \item Para probar que $P_X$ es una función de probabilidad bastará, como en temas anteriores, con que cumpla los tres axiomas
                                        de Kolmogorov:
                                        \begin{enumerate}
                                                \item[A1:]  $P_X(B) \geq 0, \thinspace \forall B \in \mathcal{B}$. 
                                                \item[A2:]  $ P_X(\R) = 1$.
                                                \item[A3:]  $ \{B_i\}_{i \in \N} \subseteq \mathcal{B}$ y $B_i \cap B_j = \emptyset, \thinspace \forall i \neq j 
                                                                \Rightarrow P_X \left( \bigcup_{i=1}^{+\infty} B_i\right) = \sum_{i=1}^{+\infty} P_X(B_i)$.
                                        \end{enumerate}
                        \end{enumerate}
\end{comment}

        \end{enumerate}



\end{document}